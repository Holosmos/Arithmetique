\documentclass{mybourbaki}
\titre{Arithmétique dans $\Z$}
\renewcommand{\vect}[1]{\boldsymbol{#1}}
\renewcommand{\div}{\mid}
\begin{document}

\section{Notions élémentaires}
\subsection{Sous-groupe de $\Z$}

\theoreme{
Tout sous-groupe de $\Z$ est de la forme $a\Z$ avec $a\in \Z$.
}{}
\demonstration{
Soit $G$ un sous-groupe de $\Z$. Si $G$ est réduit à $\ens{0}$ alors $G = 0\Z$.

Sinon, soit $x = |x|\in G$ l'élément minimal de $G$ non nul (qui existe puisque $G\dans \Z$). $x\Z \dans G$ puisque $G$ est un groupe.

Soit $y\in G$. Par division euclidienne, il existe un unique couple $(a,b)\in \Z\times\ens{0,1,\ldots,x-1}$ tel que $y =ax+b$. On a $y-ax\in G$ mais aussi $y-ax = b \in G$. Or $b<x$ donc $b=0$ et donc $y\in x\Z$.
}{}
On notera $x\Z = \vect{x}$.

\subsection{$\pgcd$ et $\ppcm$}

\definition{ 
Soient $a,b\in \Z$. On dit de manière équivalente : \[ a \div b \ssi \vect{b} \dans \vect{a} \ssi \exists c \in \Z, a = cb.\]
}{}

\definition{ Soient $a,b\in \Z$. 

On a que $\vect{a} + \vect{b}$ est un sous-groupe de $\Z$ de la forme $\vect{d}$ avec $d\in \Z$. On note $\pgcd(a,b) = d$.

$\vect{a}\inter \vect{b}$ est un sous-groupe de $\Z$ de la forme $\vect{m}$ avec $m\in \N$. On note $\ppcm(a,b) = m$.
}{}

\proposition{ 
Soient $a,b\in \Z$. On a les propositions suivantes :
\begin{enumerate}
\item $\pgcd(a,b) = d \ssi \left( \forall x\in \Z, x\div a \et x\div b \ssi x \div d \right)$ ;
\item $\ppcm(a,b) = m \ssi \left( \forall x\in \Z, a \div  x \et b \div x \ssi m \div x\right)$.
\end{enumerate}
}{}
\demonstration{ 
Soient $a,b\in \Z$.
\begin{enumerate}
\item Soit $x\in \Z$. $x\div a \et x\div b \ssi \vect{a}\dans \vect{x} \et \vect{b}\dans \vect{x}\ssi \vect{a} + \vect{b} \dans \vect{x} \ssi x \div \pgcd(a,b)$.
\item Soit $x\in \Z$. $a\div x \et b \div x \ssi \vect{x} \dans \vect{a} \et \vect{x} \dans \vect{b} \ssi \vect{x} \dans \vect{a}\inter \vect{b} \ssi \ppcm(a,b)\div x$.
\end{enumerate}
}{}

\theoreme{ 
Si $a,b\in \Z$ alors il existe $u,v\in \Z$ tels que $au + bv = \pgcd(a,b)$.

En particulier, si $a$ et $b$ sont premiers entre eux alors $au + bv = 1$.
}{Identité de Bezout}

\section{Notions modulaires}

\subsection{Passage au quotient}
\definition{ 
Soit $a\in \Z$, $\Z/\vect{a}$ est le sous-ensemble de $\Z$ obtenu par le quotient de la relation d'équivalence $\cdot \div \cdot$ de $\Z$ par $\vect{a}$.

Si $x\in \Z$ alors $\barre{x}$ est la classe d'équivalence de $x$ dans $\Z/\vect{a}$.}{}

Si $\bar{x} = \bar{y}\in \Z/\vect{a}$ alors pour $x,y$ des représentants de leurs classes respectives : \[x = y + ua\]avec $u\in \Z$.

\proposition{ 
Soit $a\in \N$, $\Z/\vect{a} = \ens{\barre{0},\barre{1},\ldots,\barre{a-1}}$.
}{}

On étend naturellement les opérations sur $\Z$ à $\Z/\vect{a}$.

\theoreme{ 
$\Z/\vect{p}$ est un groupe si, et seulement si, $p$ est premier.
}{}
\demonstration{ 
Soit $p$ premier et $\barre{x}\in \Z/\vect{p}$. Soit $x$ un représentant de $\barre{x}$ dans $\Z$.
\begin{enumerate}
\item Soit $x$ est nul et c'est l'élément neutre d'inverse lui-même ;
\item Soit $x$ est non nul et on a \[ x = ap\]avec $a\in \Z^*$. \[-ap = (-1)ap \] et donc 
\end{enumerate}
}{}

\end{document}













