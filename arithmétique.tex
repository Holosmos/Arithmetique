\documentclass{mybourbaki}
\titre{Arithmétique dans $\Z$}
\renewcommand{\vect}[1]{\boldsymbol{#1}}
\renewcommand{\div}{\mid}
\newcommand{\card}{\sharp}
\begin{document}

\section{Notions de groupes}

\subsection{Groupe abstrait}

\subsection{Groupes cycliques et monogènes}

\definition{ 
Si un groupe multiplicatif $G$ est engendré par l'un de ses éléments $g$ alors il est dit :
\begin{itemize}
\item \textit{cyclique} si $G$ est fini ;
\item \textit{monogène} sinon.
\end{itemize}
Un tel élément $g$ est un \textit{générateur} de $G$.
}{}


Une première proposition fondamentale :
\theoreme{ 
Tout groupe monogène  $G$ est isomorphe à $\Z$. Tout groupe cyclique est isomorphe à $\Z/n\Z$ pour un certain $n\in \Z$.
}{}
\demonstration{ 
Il suffit de donner de bons isomorphismes :
\begin{enumerate}
\item Si $G$ est monogène alors l'homomorphisme : \[ \chi : g^n \donne  n\] est une bijection de $G$ dans $\Z$ ;
\item si $G$ est cyclique alors l'homomorphisme $\chi$ est un isomorphisme de $G$ dans $\Z/n\Z$.
\end{enumerate}
}{}

\proposition{ 
Tout groupe monogène ou cyclique est abélien.
}{}
\demonstration{ 
En effet : \[g^{n}g^{m} = g^{n+m} = g^{m+n} = g^{m}g^{n}. \]
}{}

\proposition{ 
Si $f: G \vers H$ est homomorphisme surjectif de groupes et si $G$ est un groupe engendré par $g$ alors $H$ est engendré par $f(g)$.
}{}

\section{Notions en arithmétique}
\subsection{Sous-groupe de $\Z$}

\theoreme{
Tout sous-groupe de $\Z$ est de la forme $a\Z$ avec $a\in \Z$.
}{}
\demonstration{
Soit $G$ un sous-groupe de $\Z$. Si $G$ est réduit à $\ens{0}$ alors $G = 0\Z$.

Sinon, soit $x = |x|\in G$ l'élément minimal de $G$ non nul (qui existe puisque $G\dans \Z$). $x\Z \dans G$ puisque $G$ est un groupe.

Soit $y\in G$. Par division euclidienne, il existe un unique couple $(a,b)\in \Z\times\ens{0,1,\ldots,x-1}$ tel que $y =ax+b$. On a $y-ax\in G$ mais aussi $y-ax = b \in G$. Or $b<x$ donc $b=0$ et donc $y\in x\Z$.
}{}
On notera $x\Z = \vect{x}$.

\subsection{$\pgcd$ et $\ppcm$}

\definition{ 
Soient $a,b\in \Z$. On dit de manière équivalente que \og $a$ \textit{divise} $b$\fg{} ou : \[ a \div b \ssi \vect{b} \dans \vect{a} \ssi \exists c \in \Z, a = cb.\]
}{}

\definition{ Soient $a,b\in \Z$. 

On a que $\vect{a} + \vect{b}$ est un sous-groupe de $\Z$ de la forme $\vect{d}$ avec $d\in \Z$. On note $\pgcd(a,b) = d$.

$\vect{a}\inter \vect{b}$ est un sous-groupe de $\Z$ de la forme $\vect{m}$ avec $m\in \N$. On note $\ppcm(a,b) = m$.
}{}

\proposition{ 
Soient $a,b\in \Z$. On a les propositions suivantes :
\begin{enumerate}
\item $\pgcd(a,b) = d \ssi \left( \forall x\in \Z, x\div a \et x\div b \ssi x \div d \right)$ ;
\item $\ppcm(a,b) = m \ssi \left( \forall x\in \Z, a \div  x \et b \div x \ssi m \div x\right)$.
\end{enumerate}
}{}
\demonstration{ 
Soient $a,b\in \Z$.
\begin{enumerate}
\item Soit $x\in \Z$. $x\div a \et x\div b \ssi \vect{a}\dans \vect{x} \et \vect{b}\dans \vect{x}\ssi \vect{a} + \vect{b} \dans \vect{x} \ssi x \div \pgcd(a,b)$.
\item Soit $x\in \Z$. $a\div x \et b \div x \ssi \vect{x} \dans \vect{a} \et \vect{x} \dans \vect{b} \ssi \vect{x} \dans \vect{a}\inter \vect{b} \ssi \ppcm(a,b)\div x$.
\end{enumerate}
}{}

\theoreme{ 
Si $a,b\in \Z$ alors il existe $u,v\in \Z$ tels que $au + bv = \pgcd(a,b)$.

En particulier, si $a$ et $b$ sont premiers entre eux alors $au + bv = 1$.
}{Identité de \textsc{Bezout}}

\section{Notions modulaires}

\subsection{Passage au quotient}
\definition{ 
Soit $a\in \Z$, $\Z/\vect{a}$ est le sous-ensemble de $\Z$ obtenu par le quotient de $\Z$ par $\vect{a}$.

Si $x\in \Z$ alors $\barre{x}$ est la classe d'équivalence de $x$ dans $\Z/\vect{a}$.}{}

Si $\bar{x} = \bar{y}\in \Z/\vect{a}$ alors pour $x,y$ des représentants de leurs classes respectives : \[x = y + ua\]avec $u\in \Z$. Il arrive de le noter : \[ x \congru y \mod a.\]

\proposition{ 
Soit $a\in \N$, $\Z/\vect{a} = \ens{\barre{0},\barre{1},\ldots,\barre{a-1}}$.
}{}

On étend naturellement les opérations sur $\Z$ à $\Z/\vect{a}$ en identifiant les opérations de $\Z$ à $\Z/\vect{a}$. On pourra alors confondre $x$ et son représentant $\barre{x}$ dans $\Z/\vect{a}$.

\proposition{ 
Soient $x,y\in \Z$ :
\begin{enumerate}
\item $\barre{x+y} = \barre{x} + \barre{y}$ ;
\item $\barre{xy} = \barre{x}\cdot\barre{y}$.
\end{enumerate}
}{}
\demonstration{
Soient $x,y\in \Z$.
\begin{enumerate}
\item si $x= ua + b$ et $y = va + c$ tels que dans les divisions euclidiennes respectives alors \[ \barre{x+y} = \barre{(u+v)a + b+c} = \barre{b+c} = \barre{b} + \barre{c} = \barre{x} + \barre{y} ;\]
\item de même : \[ \barre{xy} = \barre{(ua +b)(va+c)} = \barre{a(uva + uc + bv)+bc} = \barre{bc} =\barre{b}\cdot \barre{c} = \barre{x} \cdot \barre{y}.\]
\end{enumerate}
}{}

\subsection{Inverse modulaire}

\definition{ 
L'inverse par la multiplication modulo $n\in \Z$ d'un entier $a\in \Z$ est un entier $u\in \Z$ satisfaisant à \[ a^{-1} \congru u \mod n.\] C'est-à-dire de manière équivalente : \[ au \congru 1 \mod n.\]
}{}

\proposition{
$a\in \Z$ est inversible dans $\Z/\vect{n}$ si, et seulement si, $n$ est premier avec $a$.
}{}
\demonstration{ Soient $a,u,n,m\in \Z$.
\[ au \congru 1 \mod n \ssi au = 1 +mn \ssi au - mn = 1\] ce qui revient à dire que $\pgcd(a,n) = 1$.
}{}

\theoreme{ 
$\Z/\vect{p}$ est un corps si, et seulement si, $p$ est premier.
}{}
\demonstration{ 
$\Z/\vect{p}$ est un anneau. Or $x\in \Z/\vect{p}$ est inversible si $x$ est premier avec $p$ et donc tout $x$ est inversible si $p$ est premier.
}{}

\subsection{Petit théorème de \textsc{Fermat}}

\theoreme{ 
Soient $p$ un nombre premier et $a\in \Z$. Alors : \[ a^{p}\congru a \mod p.\]
}{Petit théorème de \textsc{Fermat}}
\demonstration{ 
On procède par récurrence sur $a$ :
\begin{enumerate}
\item Pour $a= 1$ c'est vérifié.
\item Pour tout $k\in \Z$ on a : \[(k+1)^{p} \congru k^{p} + 1 \mod p. \] En effet les coefficients binomiaux à l'exceptions des premier et dernier termes disparaissent en raison d'un facteur proportionnel à $p$.
\item Si la proposition est vérifiée pour $a=k$ alors pour $a=k+1$ elle est également vérifiés en raison du résultat précédent : \[ (k+1)^{p}\congru k^{p} + 1 \congru k +1 \mod p.\]
\end{enumerate}
}{}

\paragraph{Première généralisation}
On peut aller plus loin en généralisant ce résultat :
\theoreme{ 
Soit $n>0$ et $a$ entier premier avec $n$ alors :
\[ a^{\varphi(n)} \congru 1 \mod n.\]
}{Théorème d'\textsc{Euler}}
Avec bien entendu $\varphi$ l'indicatrice d'\textsc{Euler}.

Pour démontrer ce résultat on aura besoin du théorème de \textsc{Lagrange} :
\theoreme{ 
Pour tout groupe $G$ et tout sous-groupe $H$ de $G$, l'ordre (i.e. le cardinal) de $H$ divise l'ordre de $G$ : \[\card H \div \card G. \]
}{Théorème de \textsc{Lagrange}}
\demonstration{
Le groupe $(\Z/\vect{n})^{*}$ des entiers inversibles de l'anneau $\Z/\vect{n}$ est constitué des classes d'entiers inversibles modulo $n$, i.e. premiers avec $n$. Il y en a exactement $\varphi(n)$ donc ce groupe est d'ordre $\varphi(n)$.

Puisque $a$ est premier avec $n$, $\barre{a}$ est dans le groupe $(\Z/\vect{n})^*$. $\barre{a}$ a donc un ordre dans ce groupe, disons $m$ et cet ordre divise $\varphi(n)$ tel que $mk =\varphi(n)$. On a donc : \[ a^{\varphi(n)} \congru a^{mk} \congru \left(a^{m}\right)^{k} \congru 1^{k} \congru 1 \mod n.\]
}{Théorème d'\textsc{Euler}}

\demonstration{
Le cardinal de l'ensemble $G/H$ est appelé \textit{indice} de $H$ dans $G$ et est noté $[G:H]$. 
De plus, ces classes forment une partition de $G$ et chacune d'entre elles a le même cardinal que $H$. On a alors : \[\card G = \card H \times [G:H]. \]
}{Théorème de \textsc{Lagrange}}


\paragraph{Seconde généralisation}Une seconde généralisation de ce résultat est possible. Elle est donnée par :

\theoreme{ 
Si $p$ est un nombre premier et $m$ et $n$ tels que \[m \congru n \mod p-1,\] alors pour tout $a\in \Z$ on a : \[ a^{m}\congru a^{n}\mod p.\]
}{}
\demonstration{ 
En effet, soit $a$ est divisible par $p$ et les deux membres sont égaux à $0$, soit $a$ ne l'est pas et en supposant $n>m$ : \[ a^{n-m} = \left(a^{p-1}\right)^{(n-m)/(p-1)} = 1^{(n-m)/(p-1)} = 1.\]
}{}

\end{document}













